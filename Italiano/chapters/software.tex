\section{Software esterni}
I software esterni utilizzati sono molteplici
\subsection{CMake: compilazione cross-platform automatizzata}
Cmake è un software per l'automazione dello sviluppo, riducendo al minimo il problema di creare complessi script di compilazione per differenti piattaforme e differenti architetture. Cmake inoltre, tramite comandi specifici, controlla l'esistenza di librerie e applicazioni da cui il tuo software dipende. L'importante è creare un file CMakeLists.txt in ogni cartella utilizzata dove si esplicitano librerie e software utilizzati, insieme ai codici sorgente e i vari percorsi ed opzioni aggiuntivi. Un esempio di file verrà mostrato nel capitolo del programma.
\subsection{Doxygen: estrazione automatica documentazione}
Doxygen è un software per la generazione automatica della documentazione a partire dal codice sorgente.
È il sistema di documentazione di gran lunga più utilizzato nei grandi progetti open source in C++.
Il sistema estrae la documentazione dai commenti inseriti nel codice sorgente e dalla dichiarazione delle strutture dati.
L'importante è creare un file di configurazione per gestire l'output desiderato. Un esempio di file verrà mostrato nel capitolo del programma.
\subsection{\LaTeX{}: linguaggio markup per compilazione testi}
\LaTeX{} è un linguaggio creato e pensato per redigere documenti di vario tipo. Esso è ampliamente utilizzato per documenti scientifici, libri e presentazioni. La tesina è stata interamente scritta utilizzando \LaTeX{}.
\section{Configurare Doxygen}
Per configurare doxygen mi è bastato creare il file di default tramite il comando doxygen -g doxyfile per poi modificarlo a mio piacimento e modificare dei parametri a seconda di cosa CMake trova nel computer. A seconda dei programmi installati doxygen creerà documentazione in HTML e \LaTeX{}.\\
Il file è troppo lungo per essere compreso nella tesina.
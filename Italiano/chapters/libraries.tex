\section{Librerie esterne}
Le librerie esterne utilizzate sono molteplici
\subsection{SFML: Simple and Fast Multimedia Library}
SFML è una libreria multipiattaforma che permette di utilizzare le direttive OpenGL e altri diversi moduli per gestire audio input e grafica. La libreria è scritta in C++ e mira a supportare computer anche di vecchia data.
\subsection{Thor: Estensione di SFML che utilizza C++11}
Thor è una estensione di SFML, in quanto aggiunge funzionalità molto utili, che verranno espresse nella sezione del programma, sfruttando le nuove aggiunte portate dallo standard C++11.
\subsection{TGUI: The Graphical User Interface for SFML}
Tgui è una semplice libreria che permette di creare bottoni e gestirne la funzione velocemente, molte delle sue funzionalità non vengono utilizzate nel progetto perchè dopo vari test considero la libreria poco raffinata per essere utilizzata ampliamente.